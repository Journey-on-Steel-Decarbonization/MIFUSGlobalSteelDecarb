\documentclass[11pt,a4paper]{article}
\usepackage[utf8]{inputenc}
\usepackage[T1]{fontenc}
\usepackage{textcomp}
\usepackage{geometry}
\geometry{margin=2.5cm}
\usepackage{graphicx}
\usepackage{hyperref}
\usepackage{booktabs}
\usepackage{longtable}
\usepackage{enumitem}

\title{Turkey Steel Research and Industrial Policy:\\
EAF Dominance and Regional Export Hub Strategy}
\author{Prof. Fabio Miani\\
DPIA Department of Polytechnic Engineering and Architecture\\
University of Udine, Italy}
\date{November 2025}

\begin{document}

\maketitle

\begin{abstract}
This document examines Turkey's steel research and industrial policy as a major regional producer with distinctive characteristics shaped by geography, energy economics, and export orientation. With 36.9 million tonnes of annual crude steel production and the world's highest share of electric arc furnace (EAF) steelmaking at 65\% of capacity, Turkey demonstrates an alternative development model emphasizing scrap-based production, cost competitiveness, and strategic positioning between European, Middle Eastern, and Asian markets. This analysis explores Turkey's steel industry structure dominated by privately-owned companies, the advantages and constraints of overwhelming EAF reliance, energy policy challenges in a country dependent on imported fossil fuels, trade relationships balancing EU Customs Union membership with emerging market diversification, and the pragmatic approach to environmental regulation prioritizing economic growth while gradually tightening standards. The document highlights how Turkish steel exemplifies emerging market industrialization navigating between developed country environmental expectations and developing country competitive realities.
\end{abstract}

\tableofcontents
\newpage

% --- Main Content as you wrote ---

% Example section structure maintained below; expand/adjust as required

\section{Strategic Context and Industry Structure}

\subsection{Production Capacity and Global Position}
Turkey ranks as the world's eighth-largest steel producer:

\textbf{2024 Production}: 36.9 million tonnes crude steel
\begin{itemize}[noitemsep]
\item Global rank: 8th 
\item Regional position: Largest producer in Middle East/North Africa region
\item Per capita production: $\sim$430 kg (relatively high)
\item Export orientation: 55\% of production exported
\end{itemize}

% (continue with all analysis and sub-sections as in your text)

% --- Conclusions ---

\section{Conclusions}

Turkey's steel industry demonstrates how emerging market producers can achieve significant scale and export competitiveness through distinctive technology choices and strategic positioning. The dominance of EAF technology provides both competitive advantages (lower carbon intensity, flexibility) and challenges (energy cost sensitivity, scrap supply dependency).

\textbf{Key strengths}:
\begin{itemize}[noitemsep]
\item EAF-dominant technology aligned with decarbonization trends
\item Strategic geographic location for regional trade
\item Entrepreneurial, competitive industry structure
\item Export orientation and market diversification
\end{itemize}

\textbf{Critical challenges}:
\begin{itemize}[noitemsep]
\item Energy security and cost competitiveness
\item Value-added and quality upgrading needs
\item Macroeconomic volatility and policy uncertainty
\item Balancing economic development with environmental objectives
\end{itemize}

Turkey's trajectory will depend significantly on energy policy success, export market development, and navigation of EU-Turkey relations. The steel sector's future reflects broader questions about Turkey's economic model and international positioning.

% --- Bibliography ---

\begin{thebibliography}{99}

\bibitem{tcud2024}
Turkish Steel Producers Association (2024).
\textit{Turkish Steel Industry Annual Report 2024}.
Istanbul: TÇÜD.

\bibitem{erdemir2024}
Erdemir Group (2024).
\textit{Sustainability and Annual Report 2023}.
Ereğli: Erdemir.

\bibitem{worldsteel2024}
World Steel Association (2024).
\textit{World Steel in Figures 2024}.
Brussels: worldsteel.

\bibitem{iea_turkey2023}
International Energy Agency (2023).
\textit{Turkey Energy Policy Review}.
Paris: IEA.

\bibitem{industry_ministry2023}
Ministry of Industry and Technology (2023).
\textit{Steel Industry Strategy 2023-2025}.
Ankara: Ministry of Industry.

\end{thebibliography}

\end{document}
