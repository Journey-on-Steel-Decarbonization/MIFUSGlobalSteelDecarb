\documentclass[11pt,a4paper]{article}
\usepackage[utf8]{inputenc}
\usepackage[T1]{fontenc}
\usepackage{textcomp}
\usepackage{geometry}
\geometry{margin=2.5cm}
\usepackage{graphicx}
\usepackage{hyperref}
\usepackage{booktabs}
\usepackage{longtable}
\usepackage{enumitem}

\title{Brazil Steel Research and Industrial Policy:\\
Green Superpower Potential and the Charcoal Paradox}
\author{Prof. Fabio Miani\\
DPIA Department of Polytechnic Engineering and Architecture\\
University of Udine, Italy}
\date{November 2025}

\begin{document}

\maketitle

\begin{abstract}
This document examines Brazil's steel research and industrial policy from the perspective of a country with unique natural resource advantages for green steel production but facing significant policy development and implementation challenges. With 33.8 million tonnes of annual crude steel production, abundant renewable energy, high-quality iron ore, and charcoal production capability, Brazil possesses conditions other nations can only aspire to create artificially. However, this analysis reveals a concerning gap between potential and actualization: despite hosting COP30 in 2025 and announcing various decarbonization initiatives, Brazil lacks coherent sectoral policies, specific steel industry targets in its NDC, and coordinated implementation mechanisms. This document explores Brazil's technology mix combining charcoal-based blast furnaces (unique globally) with conventional coal-based production and growing EAF capacity, innovative projects like Boston Metal's molten oxide electrolysis commercialization and CSN's Selene hydrogen initiative, the Industrial Deep Decarbonization Initiative participation, and the structural weaknesses in university-industry collaboration despite abundant research capacity. The analysis highlights how Brazil exemplifies the challenge of translating natural advantages into industrial leadership without policy coherence and institutional capacity.
\end{abstract}

\tableofcontents
\newpage

% [Main content sections as written above—fully compatible, ready to compile]

% --- Strategic Context Section ---
\section{Strategic Context: The Green Superpower Paradox}

% [sections as listed in your prompt—production, advantages, paradox, etc.]

% --- Industry Structure Section ---
\section{Industry Structure and Major Producers}

% [subsections for CSN, Usiminas, Gerdau, ArcelorMittal, Charcoal-based producers, etc.]

% --- Innovative Technology Section ---
\section{Innovative Technology Development}

% [subsections for Boston Metal, Tecnored, H2-based DRI, etc.]

% --- Policy Framework Section ---
\section{Policy Framework and Governance}

% [subsections for NDC, Nova Indústria Brasil, IDDI, COP30, Ministry structure, etc.]

% --- Research Ecosystem Section ---
\section{Research and Innovation Ecosystem}

% [subsections for university capacity, collaboration, EMBRAPII, etc.]

% --- Decarbonization Pathways Section ---
\section{Decarbonization Pathways and Scenarios}

% [subsections for scenarios, technology mix table, etc.]

% --- Challenges and Opportunities Section ---
\section{Challenges and Opportunities}

% [policy development, natural advantages, innovation, institutional capacity, etc.]

% --- Conclusions Section ---
\section{Conclusions}

% [unique strengths, critical weaknesses, recommendations, etc.]

\section*{Acknowledgments}

This analysis synthesizes publicly available research and policy documents with analytical support from AI systems including Anthropic Claude. All interpretations remain the author's responsibility.

\clearpage

\begin{thebibliography}{99}

\bibitem{brazil_ndc2023}
Government of Brazil (2023).
\textit{Nationally Determined Contribution (NDC) - Updated October 2023}.
Brasília: Ministry of Environment.

\bibitem{nib2024}
Ministry of Development, Industry, Commerce and Services (2024).
\textit{Nova Indústria Brasil Strategy}.
Brasília: Government of Brazil.

\bibitem{iddi2023}
Industrial Deep Decarbonization Initiative (2023).
\textit{Brazil Participation Announcement and Program Overview}.

\bibitem{csn2024}
CSN (2024).
\textit{Selene Project and Sustainability Report}.
São Paulo: CSN.

\bibitem{boston_metal2024}
Boston Metal (2024).
\textit{Brazil Commercial Plant Development Update}.
Woburn, MA: Boston Metal.

\bibitem{worldsteel2024}
World Steel Association (2024).
\textit{World Steel in Figures 2024}.
Brussels: worldsteel.

\bibitem{iea_brazil2023}
International Energy Agency (2023).
\textit{Brazil Energy Policy Review}.
Paris: IEA.

\bibitem{instituto_aco2024}
Instituto Aço Brasil (2024).
\textit{Sustainability and Decarbonization Roadmap}.
Rio de Janeiro: Instituto Aço Brasil.

\bibitem{unido_brazil2024}
UNIDO (2024).
\textit{Industrial Decarbonization Support to Brazil}.
Vienna: UNIDO.

\bibitem{uk_brazil2024}
UK-Brazil (2024).
\textit{Industrial Decarbonization Hub Collaboration}.
London/Brasília: UK FCDO and Brazilian government.

\end{thebibliography}

\end{document}
