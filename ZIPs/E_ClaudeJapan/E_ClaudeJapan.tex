\documentclass[11pt,a4paper]{article}
\usepackage[utf8]{inputenc}
\usepackage[margin=2.5cm]{geometry}
\usepackage{graphicx}
\usepackage{hyperref}
\usepackage{xcolor}
\usepackage{booktabs}
\usepackage{enumitem}
\usepackage{tikz}

\definecolor{japancolor}{RGB}{188,0,45}

\title{\textbf{Japan Steel Policy and Green Transformation 2024-2025}}
\author{Industrial Policy Analysis}
\date{November 2025}

\begin{document}

\maketitle

\section{Executive Summary}

Japan's steel industry, accounting for 14\% of the nation's carbon emissions and responsible for over 40\% of industrial CO\textsubscript{2} emissions, stands at a critical transformation point. As the world's third-largest steel producer and second-largest exporter, Japan has committed to achieving carbon neutrality by 2050, requiring fundamental restructuring of its steelmaking processes.

With approximately half of Japan's blast furnaces reaching end-of-life by 2030, the country faces a unique opportunity to invest in green steel technologies. The government has launched comprehensive support programs, including the Green Transformation (GX) Promotion Act and Green Innovation Fund, to facilitate the transition from traditional blast furnace-basic oxygen furnace (BF-BOF) production to electric arc furnaces (EAF) and hydrogen-based steelmaking.

This document analyzes Japan's steel policies, decarbonization roadmap, technology development initiatives, and the challenges facing the industry's transition to sustainable production.

\section{Current Steel Industry Profile}

\subsection{Production and Global Position}

\begin{itemize}[leftmargin=*]
    \item \textbf{Global Ranking:} Third-largest steel producer worldwide
    \item \textbf{Export Share:} 40\% of domestically produced steel exported (second-largest exporter globally)
    \item \textbf{Primary Markets:} Asia-Pacific region, particularly Southeast Asia
    \item \textbf{Quality Reputation:} Globally recognized for high-performance, high-quality steel materials
    \item \textbf{Energy Efficiency:} World's highest level of energy efficiency in steel production processes
\end{itemize}

\subsection{Production Technology Mix}

\begin{itemize}[leftmargin=*]
    \item \textbf{Blast Furnace (BF) Iron Production:} 76\% of total steel production
    \item \textbf{Electric Arc Furnace (EAF):} 24\% of total production
    \item \textbf{Comparison with Competitors:}
    \begin{itemize}
        \item United States: 70\% EAF
        \item European Union: 42\% EAF
        \item South Korea: 32\% EAF
        \item China: 15-20\% EAF target by 2025
    \end{itemize}
\end{itemize}

\section{Carbon Neutrality Vision and Targets}

\subsection{2050 Carbon Neutrality Commitment}

Announced in March 2021 as part of the Medium- to Long-Term Management Plan, Japan's steel industry has committed to:
\begin{itemize}[leftmargin=*]
    \item \textbf{2030 Target:} 30\% reduction in total CO\textsubscript{2} emissions compared to 2013 baseline (200 million tons)
    \item \textbf{2050 Target:} Complete carbon neutrality across the steel value chain
    \item \textbf{Transition Period:} 2024-2030 focused on EAF expansion and efficiency improvements
    \item \textbf{Post-2030:} Full-scale implementation of hydrogen-based reduction technologies
\end{itemize}

\subsection{Scope of Decarbonization}

The industry's decarbonization efforts encompass two value propositions:
\begin{enumerate}[leftmargin=*]
    \item \textbf{Product-Based Solutions:} Providing high-performance steel products that contribute to CO\textsubscript{2} reduction throughout society (downstream emissions reduction)
    \item \textbf{Process-Based Solutions:} Decarbonizing the steelmaking process itself to provide carbon-neutral steel (direct emissions reduction)
\end{enumerate}

\section{Government Policy Framework}

\subsection{Green Transformation (GX) Promotion Act}

\textbf{Energy and Manufacturing Process Transformation Support Business (2025-2029):}

Major steel companies selected for government support:
\begin{itemize}[leftmargin=*]
    \item \textbf{Nippon Steel:} 251.4 billion yen (US\$ 1.74 billion) in government support for 868.7 billion yen (US\$ 6.02 billion) EAF investment
    \item \textbf{JFE Steel:} Support for US\$ 2.2 billion EAF construction at West Japan Works
    \item \textbf{Kobe Steel:} Initial commitment reduced from US\$ 2.09 billion to US\$ 1.05 billion amid shareholder concerns
\end{itemize}

\subsection{7th Strategic Energy Plan (SEP)}

Released in late 2024, the 7th SEP outlines Japan's energy policy through 2030 with implications for steel:
\begin{itemize}[leftmargin=*]
    \item \textbf{Fossil Fuel Reduction:} From 70\% to 30-40\% of electricity supply by 2040
    \item \textbf{Renewable Energy:} Increase to 40-50\% of energy mix by 2040 (largest source)
    \item \textbf{Hydrogen Development:} Critical for steel decarbonization with planned infrastructure development
    \item \textbf{Energy-Industry Integration:} Close connection between energy and industrial policy for competitiveness
\end{itemize}

\subsection{GX2040 Vision}

Key emphasis on ``visualizing the value of green products'':
\begin{itemize}[leftmargin=*]
    \item Carbon footprint (CFP) quantification
    \item Emission reductions measurement in production
    \item Contributions to societal decarbonization
    \item Market creation for green steel products
\end{itemize}

\subsection{Act on Promoting Green Procurement}

\textbf{January 2025 Amendment:}
\begin{itemize}[leftmargin=*]
    \item Defines green products and prioritization in public procurement
    \item Includes JISF (Japan Iron and Steel Federation) Mass Balance Steel
    \item \textbf{Controversy:} Lacks explicit inclusion of scrap-based low-carbon EAF steel, causing confusion for suppliers
    \item \textbf{Clean Energy Vehicle Subsidy:} Up to 50,000 yen additional subsidy for vehicles using innovative EAF steel (FY2025)
\end{itemize}

\section{Technology Development Programs}

\subsection{NEDO Green Innovation Fund}

\textbf{Hydrogen Utilization in Iron and Steelmaking Processes Project:}

Three major steelmakers (Nippon Steel, JFE Steel, Kobe Steel) formed the Hydrogen Steelmaking Consortium with Japan Research and Development Center for Metals.

\textbf{Super COURSE50 Project:}
\begin{itemize}[leftmargin=*]
    \item \textbf{Goal:} Halve CO\textsubscript{2} emissions using heated external hydrogen
    \item \textbf{Achievement (2023):} Confirmed 33\% CO\textsubscript{2} reduction in test blast furnace (world record)
    \item \textbf{December 2024:} Achieved 43\% reduction in small-scale test furnace
    \item \textbf{FY2024 Target:} 40\% or more emissions reduction
    \item \textbf{Commercialization:} Technology establishment by 2040
\end{itemize}

\textbf{Original COURSE50 Technology:}
\begin{itemize}[leftmargin=*]
    \item Replaces carbon with hydrogen-rich gases generated in steelworks
    \item Technologies established for 50\% CO\textsubscript{2} reduction
    \item Test furnace: 12 cubic meter capacity
\end{itemize}

\subsection{Hydrogen-Based Direct Reduced Iron (DRI)}

\textbf{Nippon Steel Initiatives:}
\begin{itemize}[leftmargin=*]
    \item Research on hydrogen-based reduction using low-grade iron ore
    \item MOU with Vale (April 2022) for DR iron collaboration
    \item Supply agreement with Tenova (March 2024) for hydrogen-based DRI demonstration plant
    \item December 2024: Joint venture with Sojitz and Canadian mining company for Kami iron ore mine (DR-grade ore)
    \item Small test reduction furnace scheduled for FY2025
\end{itemize}

\textbf{JFE Steel and Kobe Steel:}
\begin{itemize}[leftmargin=*]
    \item Development of gas-based DRI technologies
    \item Future transition from natural gas to green or blue hydrogen
    \item Essential for stable DRI supply to support EAF expansion
\end{itemize}

\section{Electric Arc Furnace (EAF) Transition}

\subsection{Major Investments and Projects}

\textbf{Nippon Steel Corporation:}
\begin{itemize}[leftmargin=*]
    \item \textbf{Investment:} 868.7 billion yen (US\$ 6.02 billion) announced June 2025
    \item \textbf{Scope:} Three new EAFs at Kyushu Works plus expansion of two existing EAFs
    \item \textbf{Capacity Addition:} 2.9 million tons per year
    \item \textbf{Timeline:} Commercial operation by fiscal year 2029 (ending March 31, 2030)
    \item \textbf{Government Support:} Up to 251.4 billion yen (US\$ 1.74 billion)
    \item \textbf{Target:} EAFs to represent 8\% of production by 2030
    \item \textbf{Achievement:} Setouchi Works Hirohata Area---world's first integrated EAF for high-grade electrical steel sheets (October 2022)
    \item \textbf{R\&D:} Small EAF (10 tons capacity) at Hasaki R\&D Center operational from late FY2024
\end{itemize}

\textbf{JFE Steel Corporation:}
\begin{itemize}[leftmargin=*]
    \item \textbf{Investment:} US\$ 2.2 billion announced April 2025
    \item \textbf{Location:} West Japan Works, Kurashiki
    \item \textbf{Capacity:} 2 million tons per annum
    \item \textbf{Timeline:} Commercial launch Q1 FY2028
\end{itemize}

\textbf{Kobe Steel, Ltd.:}
\begin{itemize}[leftmargin=*]
    \item \textbf{Budget Adjustment:} Reduced decarbonization investment from US\$ 2.09 billion to US\$ 1.05 billion (FY2024-2026) in May 2025
    \item \textbf{Reason:} Shareholder concerns about competitiveness impact and costs
    \item \textbf{Commitment:} Maintains 30\% CO\textsubscript{2} reduction target by 2030
    \item \textbf{Status:} Only major player not yet committed to EAF transition
\end{itemize}

\subsection{EAF Technology Advantages}

\begin{itemize}[leftmargin=*]
    \item \textbf{Feedstock:} Uses recycled steel scrap, sharply reducing CO\textsubscript{2} emissions
    \item \textbf{Emissions Reduction:} Up to 75\% lower emissions compared to BF-BOF route
    \item \textbf{Flexibility:} Can integrate with renewable electricity for further decarbonization
    \item \textbf{Maturity:} Proven commercial technology (majority of US steel since 2002)
    \item \textbf{Recycling:} High recycling rates supporting circular economy
\end{itemize}

\subsection{Challenges in EAF Adoption}

\begin{itemize}[leftmargin=*]
    \item \textbf{Higher Operational Costs:} Electricity and raw material costs exceed BF-BOF
    \item \textbf{Scrap Availability:} Limited domestic scrap generation constrains expansion
    \item \textbf{Product Quality:} Requires high-grade scrap or DRI for premium steel grades
    \item \textbf{Infrastructure:} Need for scrap collection, sorting, and processing systems
    \item \textbf{Electricity Supply:} Requires stable, affordable, and increasingly renewable power
\end{itemize}

\section{Hydrogen Economy Development}

\subsection{Hydrogen Production and Supply}

\textbf{Current Status:}
\begin{itemize}[leftmargin=*]
    \item First 10 MW green hydrogen plant operational since 2020
    \item 7th Strategic Energy Plan targets: 15 GW electrolysis capacity by 2050
    \item Production targets: 0.3 million tons by 2030; 5-10 million tons by 2050
\end{itemize}

\textbf{Import Infrastructure:}
\begin{itemize}[leftmargin=*]
    \item Liquefied hydrogen terminal completed in Kobe Port (end of 2021)
    \item First cargo from Australia received February 2022
    \item Cost projections: \$2.25/kg by 2030; \$1.50/kg by 2050
    \item \textbf{Steel Industry Requirement:} \$0.60/kg or lower for large-scale adoption (Nippon Steel estimate)
\end{itemize}

\textbf{Hydrogen Society Promotion Law:}
\begin{itemize}[leftmargin=*]
    \item Enacted in FY2024
    \item Framework for government support being established
    \item Focus on supply chain development and cost reduction
\end{itemize}

\subsection{Steel Industry Hydrogen Requirements}

Nippon Steel alone requires several million tons of hydrogen annually for:
\begin{itemize}[leftmargin=*]
    \item Hydrogen reduction in blast furnaces (COURSE50 and Super COURSE50)
    \item Hydrogen-based DRI production
    \item Decarbonization of power generation and auxiliary processes
    \item Transition from natural gas to hydrogen in various applications
\end{itemize}

\section{Mass Balance Approach Controversy}

\subsection{Japan Iron and Steel Federation (JISF) Mass Balance Steel}

\textbf{Concept:}
\begin{itemize}[leftmargin=*]
    \item Allows companies to market steel with ``reduced emissions'' label
    \item Based on averaging emissions across production sites
    \item Does not require actual per-tonne emission reductions
    \item Version 3.1 Guidelines on Green Steel published by JISF
\end{itemize}

\textbf{Industry Examples:}
\begin{itemize}[leftmargin=*]
    \item \textbf{Kobe Steel ``Kobenable'':} Uses hot briquetted iron (HBI) from Midrex process in BFs
    \item \textbf{JFE ``JGreeX'':} Employs mass balance approach to market low-carbon steel while extending BF lifespans
\end{itemize}

\subsection{Criticism and Concerns}

\textbf{Civil Society and Environmental Groups:}
\begin{itemize}[leftmargin=*]
    \item ``Low-emission steel promoted by JISF diverges from global standards'' (June 2025 joint statement)
    \item Limited effectiveness in achieving actual emission reductions
    \item Risk of delaying real decarbonization transition
    \item Could impede production and adoption of genuine near-zero emission steel
\end{itemize}

\textbf{Policy Concerns:}
\begin{itemize}[leftmargin=*]
    \item Green Purchasing Act prioritizes mass-balanced steel over scrap-based EAF steel
    \item Confusion among suppliers using low-carbon scrap-based steel
    \item Unfair competitive disadvantage for actual low-carbon producers
    \item Counterproductive to decarbonization efforts of public and private sectors
\end{itemize}

\textbf{Need for Reform:}
\begin{itemize}[leftmargin=*]
    \item Explicit inclusion of scrap-based low-carbon steel in procurement priorities
    \item Clear definitions aligned with international standards (Responsible Steel, IEA)
    \item Focus on absolute emission reductions per tonne, not accounting methods
\end{itemize}

\section{International Standards and Market Pressures}

\subsection{Global Decarbonization Initiatives}

\textbf{SteelZero Initiative:}
\begin{itemize}[leftmargin=*]
    \item Global program bringing organizations together for net-zero steel
    \item Commitment: 50\% low-emission steel by 2030; 100\% net-zero steel by 2050
    \item Japan's participation critical given export dependence
\end{itemize}

\textbf{Responsible Steel Certification:}
\begin{itemize}[leftmargin=*]
    \item ESG-focused certification for environmental, social, and governance practices
    \item Big River Steel (Nippon Steel group, USA) achieved world's first product certification in 2024
    \item Site and product certification for non-oriented electrical steel sheets
\end{itemize}

\subsection{Carbon Border Mechanisms}

\textbf{EU Carbon Border Adjustment Mechanism (CBAM):}
\begin{itemize}[leftmargin=*]
    \item Tariffs on high-carbon goods imported from outside EU
    \item Direct impact on Japanese steel exports to Europe
    \item Requires carbon footprint disclosure and verification
\end{itemize}

\textbf{US Inflation Reduction Act:}
\begin{itemize}[leftmargin=*]
    \item Incentives for low-carbon products
    \item Potential future border adjustments
    \item Competitive pressure on high-emission steel
\end{itemize}

\textbf{Global Market Trends:}
\begin{itemize}[leftmargin=*]
    \item Shrinking markets for non-decarbonized steel
    \item Customer demand for low-carbon products
    \item Willingness to pay premium for green steel
    \item Science-based decarbonization targets throughout supply chains
\end{itemize}

\section{Blast Furnace Retirement Timeline}

\subsection{Critical Decision Point}

\begin{itemize}[leftmargin=*]
    \item \textbf{Approximately 50\%} of Japan's blast furnaces reaching end-of-life by 2030
    \item \textbf{Typical BF Lifespan:} 30-40 years with periodic relining
    \item \textbf{Relining Implications:} Extends lifetime by 15-20 years, locking in emissions
    \item \textbf{Investment Lock-in:} Early transition to EAF avoids stranded asset risks
\end{itemize}

\subsection{Transition Scenarios}

\textbf{Current Conservative Approach:}
\begin{itemize}[leftmargin=*]
    \item Limited BF-to-EAF conversions planned
    \item Some BF relining extending operational life
    \item Gradual integration of hydrogen technologies in existing BFs
    \item Risk of carbon lock-in and stranded assets
\end{itemize}

\textbf{Accelerated Transition Scenario:}
\begin{itemize}[leftmargin=*]
    \item Replace retiring BFs with EAF capacity
    \item Immediate emissions reductions of 50-75\%
    \item Alignment with global market trends
    \item Competitive advantage in green steel markets
    \item Reduced stranded asset exposure
\end{itemize}

\section{Research and Development}

\subsection{Multi-Track Technology Approach}

Japan's steel industry recognizes that no single solution exists:
\begin{itemize}[leftmargin=*]
    \item \textbf{EAF with Renewable Power:} Immediate emission reductions, mature technology
    \item \textbf{Hydrogen Reduction in BF:} Super COURSE50 for existing BF infrastructure
    \item \textbf{Hydrogen-Based DRI + EAF:} Long-term near-zero emission pathway
    \item \textbf{Carbon Capture and Storage (CCS):} Complementary technology for remaining emissions
    \item \textbf{Ammonia Co-firing:} Alternative fuel for power generation and heat processes
    \item \textbf{Biomass Integration:} Renewable carbon sources in specific applications
\end{itemize}

\subsection{Collaborative Research}

\textbf{Japanese Advanced CCS Project:}
\begin{itemize}[leftmargin=*]
    \item Seven companies conducting engineering design work (announced September 2024)
    \item Integration with steelmaking processes
    \item Concerns about cost-effectiveness and energy penalties
\end{itemize}

\textbf{International Partnerships:}
\begin{itemize}[leftmargin=*]
    \item Vale (Brazil): DR iron research collaboration
    \item Tenova: Hydrogen-based DRI demonstration plant
    \item Various technology providers for advanced materials and processes
\end{itemize}

\section{Economic and Competitive Considerations}

\subsection{Cost Challenges}

\begin{itemize}[leftmargin=*]
    \item \textbf{Hydrogen Costs:} Current prices far exceed \$0.60/kg requirement for steel industry
    \item \textbf{EAF Electricity:} Higher operational costs than BF-BOF
    \item \textbf{Capital Investment:} Massive upfront costs for technology transition
    \item \textbf{Scrap Prices:} Volatility and availability constraints
    \item \textbf{Carbon Pricing:} Uncertainty about domestic carbon pricing mechanisms
\end{itemize}

\subsection{Government Support Rationale}

The GX Promotion Act support is described as ``rather generous'' by Japanese media:
\begin{itemize}[leftmargin=*]
    \item Recognition that market forces alone insufficient for transition
    \item Strategic importance of steel industry for economy and security
    \item Competitive necessity given global decarbonization trends
    \item Support for technology leadership and export opportunities
\end{itemize}

\subsection{Shareholder and Financial Concerns}

Kobe Steel's budget reduction highlights tensions:
\begin{itemize}[leftmargin=*]
    \item Uncertainty about return on investment for green technologies
    \item Concerns about near-term competitiveness impacts
    \item Need for clearer policy frameworks and market signals
    \item Balance between environmental commitments and financial performance
\end{itemize}

\section{Market Creation and Demand-Side Measures}

\subsection{Public Procurement}

\begin{itemize}[leftmargin=*]
    \item Green Purchasing Act amendment (January 2025) prioritizing green steel
    \item Government infrastructure projects as anchor demand
    \item Risk of unintended consequences if mass balance steel prioritized over actual low-carbon steel
\end{itemize}

\subsection{Automotive Sector}

\begin{itemize}[leftmargin=*]
    \item Clean Energy Vehicle Subsidy increase (50,000 yen) for vehicles with innovative EAF steel (FY2025)
    \item Major automotive manufacturers setting supply chain decarbonization targets
    \item Demand for certified low-carbon steel materials
\end{itemize}

\subsection{Export Markets}

\begin{itemize}[leftmargin=*]
    \item 40\% of production exported, primarily to Asia
    \item Growing requirements for carbon footprint disclosure
    \item Premium pricing opportunities for certified green steel
    \item Risk of market share loss if transition too slow
\end{itemize}

\section{Comparison with Global Competitors}

\subsection{Regional Context}

\textbf{China:}
\begin{itemize}[leftmargin=*]
    \item Target: 15-20\% EAF by 2025
    \item Massive capacity adjustments and consolidation
    \item Significant government-directed transformation
\end{itemize}

\textbf{South Korea:}
\begin{itemize}[leftmargin=*]
    \item 32\% EAF share
    \item Major hydrogen economy investments
    \item Similar challenges with BF-dominant production
\end{itemize}

\textbf{United States:}
\begin{itemize}[leftmargin=*]
    \item 70\% EAF share (highest among major producers)
    \item Steel sector emits half the CO\textsubscript{2} per tonne vs. Japan
    \item Expansion of EAFs powered by renewable energy
\end{itemize}

\textbf{European Union:}
\begin{itemize}[leftmargin=*]
    \item 42\% EAF share
    \item Strict CBAM regulations driving decarbonization
    \item Significant hydrogen infrastructure investments
\end{itemize}

\subsection{Japan's Competitive Position}

\textbf{Strengths:}
\begin{itemize}[leftmargin=*]
    \item World-class energy efficiency
    \item High-quality steel reputation
    \item Advanced R\&D capabilities
    \item Strong government support programs
\end{itemize}

\textbf{Vulnerabilities:}
\begin{itemize}[leftmargin=*]
    \item Low EAF share (24\%) vs. competitors
    \item Slow pace of BF retirement/conversion
    \item Limited near-term emissions reduction from mass balance approach
    \item Risk of losing export markets to greener competitors
\end{itemize}

\section{Challenges and Barriers}

\begin{enumerate}[leftmargin=*]
    \item \textbf{Technology Readiness:} Hydrogen-based technologies not yet commercially viable at scale
    \item \textbf{Infrastructure Gaps:} Insufficient hydrogen production, storage, and distribution
    \item \textbf{Cost Competitiveness:} Green steel production significantly more expensive than conventional
    \item \textbf{Scrap Ecosystem:} Limited domestic scrap availability for EAF expansion
    \item \textbf{Policy Clarity:} Uncertainty about long-term carbon pricing and regulations
    \item \textbf{Mass Balance Issues:} Controversy over what qualifies as ``low-carbon'' or ``green'' steel
    \item \textbf{Global Coordination:} Need for international standards and market mechanisms
    \item \textbf{Timing Pressure:} BF retirement timeline creates urgency for decision-making
    \item \textbf{Financial Viability:} Massive capital requirements with uncertain returns
    \item \textbf{Supply Chain Complexity:} Coordination needed across raw materials, energy, and customers
\end{enumerate}

\section{Policy Recommendations}

\begin{enumerate}[leftmargin=*]
    \item \textbf{Accelerate EAF Transition:} Replace retiring BFs with EAF capacity rather than relining
    \item \textbf{Reform Green Procurement:} Explicitly prioritize scrap-based low-carbon steel in public purchasing
    \item \textbf{Hydrogen Cost Reduction:} Intensify efforts to achieve \$0.60/kg hydrogen price target
    \item \textbf{Scrap Infrastructure:} Develop comprehensive scrap collection, sorting, and processing systems
    \item \textbf{Standards Alignment:} Adopt international definitions for low-carbon and green steel
    \item \textbf{Carbon Pricing:} Implement clear, predictable carbon pricing to create market signals
    \item \textbf{Demand Creation:} Expand incentives beyond automotive to construction, infrastructure, and industrial sectors
    \item \textbf{Technology Funding:} Maintain robust R\&D support for breakthrough innovations
    \item \textbf{Regional Cooperation:} Coordinate with Asian partners on hydrogen supply chains and standards
    \item \textbf{Transparency Requirements:} Mandate accurate carbon footprint disclosure for all steel products
\end{enumerate}

\section{Conclusion}

Japan's steel industry faces a defining transformation over the next five to ten years. With approximately half of blast furnaces reaching end-of-life by 2030, the nation must choose between extending the life of emissions-intensive infrastructure or accelerating the transition to green steel technologies.

The government has established substantial support mechanisms through the GX Promotion Act, Green Innovation Fund, and 7th Strategic Energy Plan. Major steelmakers---particularly Nippon Steel and JFE Steel---have announced significant investments in EAF capacity, representing critical first steps toward decarbonization.

However, current plans remain conservative compared to global competitors. Japan's 24\% EAF share lags the United States (70\%), European Union (42\%), and even South Korea (32\%). The industry's reliance on mass balance accounting approaches has drawn criticism for potentially delaying genuine emission reductions.

Three technology pathways---EAF with renewable power, hydrogen reduction in blast furnaces, and hydrogen-based DRI---offer complementary solutions. Each has advantages and challenges, requiring continued R\&D, infrastructure development, and policy support.

The stakes extend beyond environmental compliance. As the world's second-largest steel exporter, Japan faces growing market pressures from carbon border mechanisms, customer demands for low-carbon products, and competition from countries with cleaner production. Early adoption of green technologies could secure premium market positions; delays risk loss of export markets and technological leadership.

Success requires bold action on multiple fronts: accelerated EAF deployment, hydrogen economy development, scrap infrastructure expansion, international standards alignment, and clear policy frameworks. The 2030 target of 30\% emissions reduction is achievable, but only with decisive shifts in investment and policy priorities.

Japan's tradition of technological innovation and quality manufacturing positions it well to lead the global green steel transition. The question is whether the industry and policymakers will move with sufficient speed and scale to capture this opportunity while meeting the urgent imperative of climate action.

\vspace{1cm}

\noindent\textit{Note: This document is based on publicly available information as of November 2025. Data sources include Japanese Government publications, NEDO, Japan Iron and Steel Federation, company reports, Global Energy Monitor, Transition Asia, SteelWatch, and international steel industry analyses.}

\end{document}