\documentclass[11pt,a4paper]{article}
\usepackage[utf8]{inputenc}
\usepackage[margin=2.5cm]{geometry}
\usepackage{graphicx}
\usepackage{hyperref}
\usepackage{xcolor}
\usepackage{booktabs}
\usepackage{enumitem}
\usepackage{tikz}

\definecolor{usacolor}{RGB}{60,59,110}

\title{\textbf{United States Steel Policy and Green Transformation 2024-2025}}
\author{Prof. Fabio Miani\textsuperscript{1}}
\date{
    \textsuperscript{1} DPIA Department of Polytechnic Engineering and Architecture \\
    University of Udine, Italy \\
    \today
}

\begin{document}

\maketitle

\section{Executive Summary}

The United States steel industry stands at a transformative juncture, driven by unprecedented federal investments through the Inflation Reduction Act (IRA), Infrastructure Investment and Jobs Act (IIJA), and CHIPS and Science Act. With approximately 18 billion dollars in investments announced between 2022 and 2025 for modernization, decarbonization, and electrification, the American steel sector is positioned to emerge as a global leader in green steel production.

The U.S. possesses significant competitive advantages: 70\% of domestic steel production already utilizes electric arc furnaces (EAF), the highest share among major producers, resulting in emissions that are 50\% lower per tonne than emissions-intensive blast furnace production. However, the industry faces critical challenges including the need to meet projected demand increases of 39.7 million tonnes by 2030 driven by clean energy infrastructure, renewable energy deployment, and domestic manufacturing reshoring.

This document analyzes the comprehensive policy framework supporting the steel industry transformation, major investment programs, technology development initiatives, market dynamics, and the path toward establishing the United States as the competitive leader in low-carbon steel production.

\section{Current Industry Profile}

\subsection{Production Capacity and Technology Mix}

\begin{itemize}[leftmargin=*]
    \item \textbf{Annual Production:} Approximately 80-85 million tonnes crude steel (2024)
    \item \textbf{Electric Arc Furnace (EAF) Share:} 70\% of production (highest among major producers)
    \item \textbf{Blast Furnace-Basic Oxygen Furnace (BF-BOF):} 30\% of production
    \item \textbf{Environmental Performance:} Releases 50\% less CO2 per tonne than most major competitors
    \item \textbf{Energy Efficiency:} Among highest globally, though significant improvement potential remains
    \item \textbf{Import Dependency:} Approximately 25\% of domestic steel consumption is imported
\end{itemize}

\subsection{Regional Distribution}

Steel production concentrated in key regions:
\begin{itemize}[leftmargin=*]
    \item \textbf{Great Lakes Region:} Michigan, Ohio, Indiana, Pennsylvania (legacy integrated mills and EAF facilities)
    \item \textbf{Southern States:} Arkansas, Alabama, Texas (growing EAF mini-mill presence)
    \item \textbf{Mid-Atlantic:} Pennsylvania, Maryland (historical production centers undergoing modernization)
\end{itemize}

The concentration of iron ore reserves in Michigan and Minnesota positions the Great Lakes as prime location for greenfield hydrogen-based direct reduced iron (DRI) facilities.

\section{Federal Policy Framework}

\subsection{Inflation Reduction Act (IRA)}

Enacted in August 2022, the IRA allocates 370 billion dollars in tax incentives and funding for clean energy transition:

\textbf{Steel Industry Impacts:}
\begin{itemize}[leftmargin=*]
    \item \textbf{Demand Generation:} Expected to create 39.7 million tonnes of new steel demand through 2030
    \item \textbf{Domestic Content Bonus:} Enhanced tax credits for projects using American-made steel and iron
    \item \textbf{Manufacturing Tax Credits:} Production tax credits for clean energy equipment manufacturing
    \item \textbf{Clean Hydrogen Production Tax Credit (45V):} Up to 3 dollars per kg for hydrogen produced with renewable energy
    \item \textbf{Carbon Capture Tax Credit (45Q):} Enhanced credits for industrial carbon capture projects
\end{itemize}

\textbf{Domestic Content Requirements:}
\begin{itemize}[leftmargin=*]
    \item All manufacturing processes for steel and iron components must occur in the United States
    \item Predominantly iron or steel products must be 95\% U.S.-made
    \item Other manufactured products must meet escalating thresholds: 60\% (2022), 65\% (2024), 75\% (2029)
    \item Strong incentive for domestic sourcing to maximize federal funding eligibility
\end{itemize}

\textbf{Steel Demand Projections by Sector (2022-2030):}
\begin{itemize}[leftmargin=*]
    \item Wind Energy: Substantial increase for turbine towers and foundations
    \item Solar Energy: Mounting structures and tracking systems
    \item Electric Vehicles: Automotive steel for battery electric vehicles
    \item Grid Infrastructure: Transmission towers, substations, transformers
    \item Hydrogen Infrastructure: Pipelines, storage, production facilities
    \item Manufacturing Reshoring: Industrial facilities and equipment
\end{itemize}

\subsection{Infrastructure Investment and Jobs Act (IIJA)}

Passed in November 2021, IIJA provides 1 trillion dollars for infrastructure renewal:

\textbf{Steel-Intensive Infrastructure:}
\begin{itemize}[leftmargin=*]
    \item Roads, highways, and bridges rehabilitation
    \item Public transit and rail systems expansion
    \item Port and airport modernization
    \item Water infrastructure and wastewater systems
    \item Broadband deployment infrastructure
\end{itemize}

\textbf{Industry Impact:}
\begin{itemize}[leftmargin=*]
    \item American Iron and Steel Institute (AISI) estimates 50,000 net tons of steel required per 1 billion dollars infrastructure spending
    \item Estimated multi-year sustained demand for structural steel, reinforcing bars, and specialty products
    \item Buy America provisions ensure domestic steel utilization in federally funded projects
\end{itemize}

\subsection{CHIPS and Science Act}

Enacted August 2022, CHIPS Act allocates 52 billion dollars for semiconductor manufacturing:

\textbf{Steel Demand Generation:}
\begin{itemize}[leftmargin=*]
    \item Construction of semiconductor fabrication facilities (fabs) across multiple states
    \item Steel-intensive cleanroom construction and support infrastructure
    \item Specialized electrical steel for precision power systems
    \item High-purity gases and chemicals distribution infrastructure
\end{itemize}

\subsection{Buy Clean Initiative}

Launched in 2021, Buy Clean leverages federal procurement to drive low-carbon materials adoption:

\textbf{Program Structure:}
\begin{itemize}[leftmargin=*]
    \item \textbf{EPA Role:} Developing first-ever climate performance standards for construction materials
    \item \textbf{Funding:} 4.5 billion dollars for General Services Administration, Department of Transportation, and EPA
    \item \textbf{Covered Materials:} Steel, concrete, asphalt, flat glass
    \item \textbf{Federal Demand:} Approximately 9 million tonnes annual steel demand by 2030
\end{itemize}

\textbf{Implementation Challenges:}
\begin{itemize}[leftmargin=*]
    \item Defining appropriate emissions thresholds without causing supply disruptions
    \item Balancing near-term emission reductions with long-term deep decarbonization pathways
    \item Avoiding carbon leakage by driving domestic production offshore
    \item Creating pathway for green steel commercialization while maintaining industry competitiveness
\end{itemize}

\textbf{State-Level Programs:}
\begin{itemize}[leftmargin=*]
    \item California developing comprehensive Buy Clean program
    \item Minnesota implementing Buy Clean study and pilot program for steel rebar and structural steel
    \item Other states considering similar procurement policies
\end{itemize}

\section{Steel Decarbonization Funding}

\subsection{DOE Office of Clean Energy Demonstrations (OCED)}

\textbf{March 2024 Awards:} 1.5 billion dollars for six iron and steel decarbonization projects funded by IRA and IIJA

\textbf{Expected Impact:}
\begin{itemize}[leftmargin=*]
    \item Avoid 2.5 million metric tonnes of CO2 emissions annually
    \item Equivalent to 747 wind turbines or 4\% of domestic iron and steel emissions
    \item Demonstration of commercial-scale green steel technologies
    \item Technology proving ground for hydrogen-based DRI and advanced EAF systems
\end{itemize}

\textbf{Technology Categories Supported:}
\begin{itemize}[leftmargin=*]
    \item Hydrogen-based direct reduced iron production
    \item Advanced electric arc furnace technologies
    \item Carbon capture, utilization, and storage (CCUS)
    \item Electrification of heating processes
    \item Energy efficiency improvements
    \item Alternative fuel integration
\end{itemize}

\subsection{DOE Advanced Manufacturing Offices}

\textbf{Advanced Materials and Manufacturing Technologies Office:}
\begin{itemize}[leftmargin=*]
    \item FY2025 Budget Request: Approximately 500 million dollars
    \item Focus on breakthrough material processing technologies
    \item Steel-specific research on hydrogen reduction, electrolysis-based production
\end{itemize}

\textbf{Industrial Efficiency and Decarbonization Office:}
\begin{itemize}[leftmargin=*]
    \item Process optimization and energy efficiency research
    \item Cross-cutting technologies applicable to steel manufacturing
    \item Waste heat recovery and industrial electrification
\end{itemize}

\textbf{Office of Manufacturing and Energy Supply Chains:}
\begin{itemize}[leftmargin=*]
    \item FY2025 Request: 113 million dollars
    \item Supply chain resilience for critical materials
    \item Coordination with steel industry on raw material security
\end{itemize}

\subsection{Recommended Budget Increases}

Industry advocates and policy analysts recommend substantial budget increases:
\begin{itemize}[leftmargin=*]
    \item Advanced Materials and Manufacturing: Expansion beyond 500 million dollars request
    \item Office of Clean Energy Demonstrations: Increase from 180 million dollars FY2025 request
    \item Rationale: Ensure deployment pace matches innovation research output
    \item Focus: Multiple 100\% clean steel and iron plants in legacy communities with union labor
\end{itemize}

\section{Major Industry Players and Investments}

\subsection{Nucor Corporation}

\textbf{Profile:}
\begin{itemize}[leftmargin=*]
    \item Largest steel producer in the United States
    \item Primarily EAF-based mini-mill operator
    \item Market leader in scrap-based steelmaking
\end{itemize}

\textbf{Decarbonization Commitments:}
\begin{itemize}[leftmargin=*]
    \item \textbf{November 2023:} Announced net-zero science-based GHG targets for 2050
    \item \textbf{2030 Interim Target:} Significant reduction in Scopes 1, 2, and 3 emissions
    \item Targets defined by Global Steel Climate Council (GSCC) methodology
    \item Focus on renewable electricity procurement and energy efficiency
\end{itemize}

\textbf{Strategic Positioning:}
\begin{itemize}[leftmargin=*]
    \item Representing over 70\% of American steel capacity using recycled steel
    \item Positioned to benefit from IRA recognition of low-emissions EAF steelmaking
    \item Investment in circular economy and scrap processing infrastructure
\end{itemize}

\subsection{United States Steel Corporation (U.S. Steel)}

\textbf{Modernization Strategy:}
\begin{itemize}[leftmargin=*]
    \item \textbf{2050 Net-Zero Roadmap:} Comprehensive decarbonization plan
    \item \textbf{Technology Transition:} Expanding EAF-based mini-mill capacity while modernizing integrated operations
    \item \textbf{DRI/HBI Development:} Plans to adopt direct reduced iron/hot briquetted iron technology with natural gas, transitioning to hydrogen
\end{itemize}

\textbf{Big River 2 Mini Mill:}
\begin{itemize}[leftmargin=*]
    \item \textbf{Location:} Osceola, Arkansas
    \item \textbf{Investment:} 3.2 billion dollars (board-approved capital increase)
    \item \textbf{Capacity:} 3 million metric tonnes annually
    \item \textbf{Timeline:} Commercial operation second half 2024
    \item \textbf{Emissions Reduction:} 10-60\% decrease against 2018 baseline depending on deployment scale
    \item \textbf{Technology:} State-of-the-art EAF with capability for high-quality flat-rolled products
\end{itemize}

\textbf{Nippon Steel Acquisition (Blocked):}
\begin{itemize}[leftmargin=*]
    \item Biden administration blocked proposed acquisition by Nippon Steel in early 2025
    \item Decision based on national security considerations and economic nationalism concerns
    \item Reflects broader trend toward protectionism in strategic industries
    \item U.S. Steel continues as independent American company
\end{itemize}

\subsection{Cleveland-Cliffs}

\textbf{Integrated Steel Operations:}
\begin{itemize}[leftmargin=*]
    \item Operates both integrated mills (BF-BOF) and EAF facilities
    \item Vertical integration including iron ore mining in Michigan and Minnesota
    \item Strategic positioning with access to domestic iron ore reserves
\end{itemize}

\textbf{Modernization Investments:}
\begin{itemize}[leftmargin=*]
    \item Upgrading existing facilities with efficiency improvements
    \item Evaluating pathways for decarbonization of blast furnace operations
    \item Potential hydrogen integration in ironmaking processes
\end{itemize}

\subsection{Steel Dynamics Inc. (SDI)}

\textbf{EAF Specialization:}
\begin{itemize}[leftmargin=*]
    \item Third-largest U.S. steel producer
    \item Exclusively EAF-based production
    \item Focus on flat-rolled and long products
\end{itemize}

\textbf{Growth Strategy:}
\begin{itemize}[leftmargin=*]
    \item Capacity expansion to meet infrastructure demand
    \item Investment in advanced EAF technologies
    \item Renewable energy procurement for power-intensive operations
\end{itemize}

\section{Green Steel Technologies and Pathways}

\subsection{Electric Arc Furnace (EAF) Dominance}

\textbf{Current Status:}
\begin{itemize}[leftmargin=*]
    \item 70\% of U.S. steel production from EAF (highest globally among major producers)
    \item Majority of steel since 2002 produced via EAF route
    \item Established infrastructure and operational expertise
\end{itemize}

\textbf{Advantages:}
\begin{itemize}[leftmargin=*]
    \item Uses 100\% recycled steel scrap, dramatically reducing emissions
    \item Flexible capacity to respond to demand fluctuations
    \item Shorter construction timelines compared to integrated mills
    \item Compatible with renewable electricity integration
    \item Lower capital investment requirements
\end{itemize}

\textbf{Emissions Profile:}
\begin{itemize}[leftmargin=*]
    \item Current: Approximately 0.4-0.6 tonnes CO2 per tonne steel (scrap-based)
    \item With Renewable Electricity: Near-zero emissions achievable
    \item Comparison: BF-BOF produces 1.8-2.3 tonnes CO2 per tonne steel
\end{itemize}

\subsection{Hydrogen-Based Direct Reduced Iron (DRI)}

\textbf{Technology Overview:}
\begin{itemize}[leftmargin=*]
    \item Uses hydrogen (or natural gas transitioning to hydrogen) to reduce iron ore
    \item Produces DRI or hot briquetted iron (HBI) for EAF feedstock
    \item Enables high-quality primary steel production without blast furnaces
    \item Essential for products requiring virgin iron rather than scrap
\end{itemize}

\textbf{Development Status:}
\begin{itemize}[leftmargin=*]
    \item Pilot and demonstration projects underway with DOE funding
    \item Natural gas-based DRI commercial technology (transition fuel)
    \item Green hydrogen-based DRI under development for zero-emission pathway
    \item Target: Commercial deployment by 2030 with favorable hydrogen economics
\end{itemize}

\textbf{Cost Projections (Boston Consulting Group Analysis):}
\begin{itemize}[leftmargin=*]
    \item \textbf{Green-Powered EAF (90\% scrap, 10\% DRI):} 385 dollars per metric tonne by 2030
    \item \textbf{Hydrogen DRI + Green-Powered EAF (80\% hot metal, 20\% scrap):} 560 dollars per metric tonne (with IRA tax credits)
    \item \textbf{Competitive Advantage:} Lower costs than European competitors due to IRA subsidies
    \item \textbf{Germany Comparison:} 390 and 640 dollars per metric tonne respectively
\end{itemize}

\subsection{Molten Oxide Electrolysis (MOE)}

\textbf{Technology Description:}
\begin{itemize}[leftmargin=*]
    \item Electrochemical reduction of iron ore using electricity
    \item Direct conversion without carbon-based reductants
    \item Produces liquid iron for continuous casting or EAF feeding
    \item Potentially lowest emissions pathway if powered by renewables
\end{itemize}

\textbf{Development Status:}
\begin{itemize}[leftmargin=*]
    \item Pre-commercial stage with pilot demonstrations
    \item Significant technical challenges remain
    \item Long-term potential for game-changing technology
    \item Timeline: Post-2030 commercial viability
\end{itemize}

\subsection{Carbon Capture, Utilization, and Storage (CCUS)}

\textbf{Application to Steel:}
\begin{itemize}[leftmargin=*]
    \item Capture CO2 from blast furnace operations
    \item Applicable to existing integrated mills
    \item Bridge technology during transition period
    \item Enhanced 45Q tax credits under IRA
\end{itemize}

\textbf{Limitations:}
\begin{itemize}[leftmargin=*]
    \item High energy penalty and operating costs
    \item Does not eliminate emissions, only captures them
    \item Requires CO2 transport and storage infrastructure
    \item Less economically attractive than transitioning to EAF or hydrogen-based routes
\end{itemize}

\section{1.5 Degree Celsius Alignment Scenario}

\subsection{Emissions Reduction Requirements}

\textbf{Current Trajectory (Business-as-Usual):}
\begin{itemize}[leftmargin=*]
    \item Power demand: 11 TWh per year (2025) increasing to 15 TWh per year (2050)
    \item Emissions: Rising 33\% to reach 83 million metric tonnes CO2 annually by 2050
    \item Equivalent to 221 fossil gas power plants annual emissions
\end{itemize}

\textbf{1.5C-Aligned Pathway:}
\begin{itemize}[leftmargin=*]
    \item \textbf{Technology Mix Required:} Molten Oxide Electrolysis-EAF, Electrowinning-EAF, Hydrogen-DRI-EAF, and modernized BF-BOF with CCUS
    \item \textbf{Clean Electricity Requirement:} 174 TWh annually by 2050
    \item \textbf{Renewable Capacity:} Approximately 28 GW wind and solar, 53 GW battery storage
    \item \textbf{Near-Zero Emission Plants:} 8\% of U.S. production (9.4 million tonnes) must come from near-zero emission ore-based plants by 2030
\end{itemize}

\subsection{Grid and Infrastructure Requirements}

\textbf{Electricity Infrastructure:}
\begin{itemize}[leftmargin=*]
    \item 174 TWh nearly equals Illinois' total 2023 electricity output
    \item Requires massive grid expansion and modernization
    \item Long-distance transmission to connect renewables to steel communities
    \item Advanced grid technologies for stability and reliability
\end{itemize}

\textbf{Interconnection Queue Challenges:}
\begin{itemize}[leftmargin=*]
    \item PJM (Mid-Atlantic): 3,042 active projects, 2-year backup in queue
    \item MISO (Midwest): 1,734 active projects, similar delays
    \item Urgent need for streamlined interconnection processes
    \item Regional coordination essential for transmission planning
\end{itemize}

\section{Raw Material Security and Supply Chains}

\subsection{Iron Ore}

\textbf{Domestic Production:}
\begin{itemize}[leftmargin=*]
    \item Major reserves in Michigan (Marquette Range) and Minnesota (Mesabi Range)
    \item Historic production centers with existing infrastructure
    \item High-grade ore suitable for direct reduction processes
    \item Cleveland-Cliffs major domestic producer with vertical integration
\end{itemize}

\textbf{Strategic Considerations:}
\begin{itemize}[leftmargin=*]
    \item U.S. does not currently designate iron ore as critical mineral (adequate domestic supply)
    \item Contrast with EU, Japan, Korea, India treating iron ore as critical
    \item Growing focus on high-purity DR-grade ore for hydrogen-based steelmaking
    \item Potential need to expand domestic beneficiation capacity
\end{itemize}

\subsection{Steel Scrap}

\textbf{Availability:}
\begin{itemize}[leftmargin=*]
    \item Robust domestic scrap generation and processing infrastructure
    \item Mature collection, sorting, and trading networks
    \item High-quality scrap supports premium EAF steel production
    \item Circular economy advantages with steel's infinite recyclability
\end{itemize}

\textbf{Demand-Supply Balance:}
\begin{itemize}[leftmargin=*]
    \item Current supply adequate for 70\% EAF share
    \item IRA-driven demand increase may strain high-grade scrap availability
    \item Need for continued investment in scrap processing technologies
    \item Quality sorting critical for advanced steel grades
\end{itemize}

\subsection{Hydrogen Supply}

\textbf{Production Pathways:}
\begin{itemize}[leftmargin=*]
    \item \textbf{Green Hydrogen:} Electrolysis using renewable electricity (IRA 45V credit up to 3 dollars per kg)
    \item \textbf{Blue Hydrogen:} Natural gas with carbon capture (lower 45V credit tier)
    \item \textbf{Current Status:} Limited commercial-scale production, rapidly expanding
    \item \textbf{Cost Target:} Below 2 dollars per kg for steel industry viability
\end{itemize}

\textbf{Infrastructure Development:}
\begin{itemize}[leftmargin=*]
    \item Regional hydrogen hubs supported by DOE funding
    \item Pipeline and storage infrastructure investment needed
    \item Co-location of hydrogen production with steel facilities under consideration
    \item Gulf Coast, Great Lakes, and Appalachian regions key development areas
\end{itemize}

\section{Market Dynamics and Competitive Position}

\subsection{Demand Growth Projections}

\textbf{IRA and IIJA Impact:}
\begin{itemize}[leftmargin=*]
    \item 39.7 million tonnes new demand from now to 2030
    \item Distributed across renewable energy, EV supply chains, grid infrastructure, and reshored manufacturing
    \item Estimated 9 million tonnes annual federal government purchases by 2030
    \item Sustained long-term demand from clean energy transition
\end{itemize}

\textbf{Capacity Utilization:}
\begin{itemize}[leftmargin=*]
    \item Current industry operating below full capacity
    \item Sufficient existing capacity to absorb initial demand increase
    \item Strategic capacity additions targeting high-growth segments
    \item Focus on electrical steel for transformers and EVs
\end{itemize}

\subsection{Import Competition and Trade Policy}

\textbf{Section 232 Tariffs:}
\begin{itemize}[leftmargin=*]
    \item Trump administration imposed 25\% tariff on steel imports (2018)
    \item Biden administration maintained tariffs with modifications
    \item EU tensions over tariff structure and carbon border mechanisms
    \item Ongoing debates about balancing protection with supply chain needs
\end{itemize}

\textbf{Global Arrangement on Sustainable Steel and Aluminum (GASSA):}
\begin{itemize}[leftmargin=*]
    \item Negotiations with EU to establish carbon-based steel club
    \item Deadline extensions due to fundamental disagreements
    \item U.S. proposal: Common external tariff using Section 232 framework
    \item EU proposal: Carbon Border Adjustment Mechanism (CBAM) approach
    \item Key tensions: WTO compatibility, decarbonization approaches, third-country treatment
\end{itemize}

\textbf{China Competition:}
\begin{itemize}[leftmargin=*]
    \item Chinese steel production approximately 10 times U.S. capacity
    \item Concerns about subsidized overcapacity and dumping
    \item U.S. tariffs and anti-dumping measures in place
    \item Growing Chinese focus on decarbonization may level playing field
\end{itemize}

\subsection{Premium Green Steel Market}

\textbf{Corporate Procurement:}
\begin{itemize}[leftmargin=*]
    \item Major corporations setting supply chain decarbonization targets
    \item Willingness to pay premium for certified low-carbon steel
    \item Automotive, construction, appliance manufacturers leading demand
    \item Science-based targets (SBTi) driving procurement shifts
\end{itemize}

\textbf{Sustainable Steel Purchasing Platform:}
\begin{itemize}[leftmargin=*]
    \item Collaborative procurement initiative aggregating buyer demand
    \item Goal: Support investment in at least three low-emission facilities (2 million tonnes per year each) by 2030
    \item Offtake agreements providing revenue certainty for green steel investments
    \item Technical partners defining product-level parameters and certification
\end{itemize}

\textbf{Market Evolution:}
\begin{itemize}[leftmargin=*]
    \item Green premium narrowing due to IRA incentives and technology improvements
    \item Potential price parity or advantage by 2030 in certain segments
    \item First-mover advantages for companies establishing green credentials
    \item Risk of market share loss for slow-transitioning producers
\end{itemize}

\section{Challenges and Barriers}

\begin{enumerate}[leftmargin=*]
    \item \textbf{Hydrogen Economics:} Green hydrogen costs must fall below 2 dollars per kg for steel industry viability; current costs significantly higher
    \item \textbf{Electricity Grid Capacity:} Massive renewable energy and transmission expansion needed; interconnection queues creating bottlenecks
    \item \item \textbf{Capital Requirements:} Multi-billion dollar investments needed for new technologies; uncertainty about returns
    \item \textbf{Policy Uncertainty:} Potential changes to IRA and climate policies under future administrations; Trump administration signals possible funding cuts
    \item \textbf{Technology Readiness:} Some pathways (MOE, hydrogen-DRI) not yet commercially proven at scale
    \item \textbf{Workforce Transition:} Need for retraining and new skills in green steel technologies; community impacts in legacy steel regions
    \item \textbf{Scrap Quality and Availability:} High-grade scrap constraints for premium steel production; investment needed in sorting technologies
    \item \textbf{International Coordination:} Lack of global standards for green steel definition and carbon accounting
    \item \textbf{Trade Policy Complexity:} Balancing protectionism with supply chain needs and climate goals
    \item \textbf{Legacy Asset Stranding:} Risk of premature retirement of existing facilities; community economic impacts
\end{enumerate}

\section{State-Level Initiatives}

\subsection{Great Lakes States}

\textbf{Minnesota:}
\begin{itemize}[leftmargin=*]
    \item Energy and Climate Omnibus Bill includes Buy Clean study and pilot program
    \item Focus on steel rebar and structural steel
    \item Data compilation on embodied carbon in building materials
    \item Prioritizing low-carbon steel in state procurement
\end{itemize}

\textbf{Ohio and Pennsylvania:}
\begin{itemize}[leftmargin=*]
    \item Home to electrical steel production facilities (only two in U.S.)
    \item Critical for transformer and EV manufacturing
    \item State support for manufacturing modernization
    \item Workforce development programs for advanced steel production
\end{itemize}

\subsection{California}

\textbf{Buy Clean California:}
\begin{itemize}[leftmargin=*]
    \item Nation's most comprehensive state-level Buy Clean program
    \item Environmental Product Declarations (EPDs) required for major projects
    \item Emissions thresholds for eligible materials
    \item Model for other states developing similar programs
\end{itemize}

\subsection{Southern States}

\textbf{Arkansas, Alabama, Texas:}
\begin{itemize}[leftmargin=*]
    \item Attracting new EAF mini-mill investments
    \item Business-friendly regulatory environment
    \item Proximity to raw materials and markets
    \item Growing clean energy resources (wind, solar)
\end{itemize}

\section{Employment and Community Transition}

\subsection{Workforce Considerations}

\textbf{Current Employment:}
\begin{itemize}[leftmargin=*]
    \item Direct steel industry employment: Approximately 140,000 workers
    \item Indirect and induced employment: 500,000+ across supply chains
    \item High-wage manufacturing jobs critical to regional economies
    \item Union representation significant in legacy steel regions
\end{itemize}

\textbf{Transition Support:}
\begin{itemize}[leftmargin=*]
    \item Federal funding recommendations for deployment in legacy communities with union labor
    \item Retraining programs for new green steel technologies
    \item Community college partnerships for workforce development
    \item Apprenticeship programs combining classroom and on-the-job training
\end{itemize}

\subsection{Legacy Steel Communities}

\textbf{Economic Revitalization:}
\begin{itemize}[leftmargin=*]
    \item Green steel investments opportunity for regional renewal
    \item Leveraging existing infrastructure and skilled workforce
    \item Preventing community decline from facility closures
    \item Creating new industries around clean energy manufacturing
\end{itemize}

\textbf{Just Transition Principles:}
\begin{itemize}[leftmargin=*]
    \item Worker protections and reemployment guarantees
    \item Community input in transition planning
    \item Investment in local infrastructure and services
    \item Environmental remediation of legacy industrial sites
\end{itemize}

\section{Policy Recommendations}

\begin{enumerate}[leftmargin=*]
    \item \textbf{Accelerate Hydrogen Cost Reduction:} Increase support for green hydrogen production to achieve below 2 dollars per kg; expedite regional hydrogen hub development
    \item \textbf{Grid Modernization:} Streamline interconnection processes; invest in transmission infrastructure connecting renewable resources to steel production centers
    \item \textbf{Expand Buy Clean:} Implement robust federal procurement standards for low-carbon steel; extend to all agencies and project types
    \item \textbf{Increase DOE Funding:} Expand budgets for Office of Clean Energy Demonstrations and Advanced Manufacturing offices to match innovation pipeline
    \item \textbf{Workforce Development:} Create comprehensive training programs in partnership with community colleges and unions; prioritize deployment in legacy steel communities
    \item \textbf{International Coordination:} Establish common definitions and standards for green steel; resolve GASSA negotiations with EU on mutually beneficial terms
    \item \textbf{Technology Deployment:} Target multiple commercial-scale 100\% clean steel plants by 2030; provide financial de-risking mechanisms
    \item \textbf{Scrap Infrastructure:} Invest in advanced scrap sorting and processing technologies to support quality requirements for high-grade steel
    \item \textbf{Carbon Pricing:} Consider complementary carbon pricing mechanisms to create long-term market signals for decarbonization
    \item \textbf{Trade Policy Coherence:} Align Section 232 tariffs with climate goals; prevent carbon leakage while supporting domestic production
    \item \textbf{Regional Planning:} Coordinate federal, state, and local efforts for integrated steel-energy-infrastructure development
\end{enumerate}

\section{Conclusion}

The United States steel industry stands at a pivotal moment with unprecedented opportunities to establish global leadership in green steel production. The convergence of federal policy support through the IRA, IIJA, and CHIPS Act, combined with the industry's existing 70\% EAF share, positions America uniquely among major steel-producing nations.

The IRA's projected generation of 39.7 million tonnes of new steel demand through 2030 provides market certainty for major investments in decarbonization technologies. With appropriate implementation of Buy Clean procurement standards and continued expansion of DOE funding programs, the pathway to multiple commercial-scale near-zero emission steel plants is achievable within this decade.

Critical success factors include: achieving green hydrogen costs below 2 dollars per kg, massive expansion of renewable electricity generation and transmission infrastructure, streamlined grid interconnection processes, and sustained policy support across administrations. The technical pathways are clear---EAF with renewable power, hydrogen-based DRI, and potentially molten oxide electrolysis---with cost projections showing U.S. competitiveness against global competitors by 2030.

However, significant challenges remain. The 174 TWh annual clean electricity requirement by 2050 necessitates unprecedented grid expansion. Hydrogen infrastructure must scale rapidly. Existing interconnection queues threaten to delay critical renewable energy projects. Policy uncertainty, particularly regarding IRA implementation continuity, creates investment hesitation.

The stakes extend beyond environmental compliance. America's steel industry can either lead the global transition and capture premium green steel markets, or risk losing competitiveness to nations making bolder investments. The integrated steel sector must modernize or face obsolescence as customers demand low-carbon materials and carbon border mechanisms proliferate.

Regional implications are profound. Legacy steel communities in the Great Lakes and Mid-Atlantic face either revitalization through green steel investments or continued decline. The promise of high-wage union manufacturing jobs in clean industries can anchor community prosperity, but only with intentional policy support and workforce transition programs.

International dimensions matter critically. Resolution of GASSA negotiations with the EU will shape global green steel markets. China's massive steel capacity and growing decarbonization focus present both competitive threats and opportunities for technology cooperation. Developing coherent trade policies that protect domestic industry while advancing climate goals requires diplomatic sophistication.

The American steel industry's transformation over the next five to ten years will determine whether the United States becomes the competitive leader in low-carbon steel or cedes this strategic sector to international competitors. With existing EAF dominance, abundant renewable energy resources, technological innovation capacity, and substantial federal funding, success is achievable. The question is whether policymakers, industry leaders, and communities will execute with sufficient speed, scale, and coordination to capture this historic opportunity.

\vspace{1cm}

\noindent\textit{Note: This document is based on publicly available information as of November 2025. Data sources include U.S. Department of Energy, Environmental Protection Agency, American Iron and Steel Institute, company reports, Global Energy Monitor, RMI, Boston Consulting Group analyses, and congressional testimony.}

\end{document}