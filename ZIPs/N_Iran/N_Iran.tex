\documentclass[11pt,a4paper]{article}
\usepackage[utf8]{inputenc}
\usepackage[T1]{fontenc}
\usepackage{textcomp}
\usepackage{geometry}
\geometry{margin=2.5cm}
\usepackage{graphicx}
\usepackage{hyperref}
\usepackage{booktabs}
\usepackage{longtable}
\usepackage{enumitem}

\title{Iran Steel Research and Industrial Policy:\\
Sanctions, Self-Sufficiency, and Natural Gas Advantages}
\author{Prof. Fabio Miani\\
DPIA Department of Polytechnic Engineering and Architecture\\
University of Udine, Italy}
\date{November 2025}

\begin{document}

\maketitle

\begin{abstract}
This document examines Iran's steel research and industrial policy under unique constraints of comprehensive international sanctions while possessing significant natural resource advantages. With 31.4 million tonnes of annual crude steel production, ranking 10th globally, Iran demonstrates how a middle-income country can develop substantial steelmaking capacity despite isolation from Western technology and markets. The analysis explores Iran's distinctive reliance on natural gas-based direct reduced iron (DRI) production, representing the world's largest DRI capacity, the sanctions-driven imperative for technological self-sufficiency and domestic equipment manufacturing, state-directed industrial policy through entities like IMIDRO coordinating a mixed state-private sector, and the paradox of abundant natural gas enabling cleaner steelmaking pathways while carbon emissions remain peripheral concerns given geopolitical priorities. The document highlights how Iranian steel exemplifies autarkic industrialization driven by resource endowments and external constraints rather than climate policy, offering insights into technology development trajectories when conventional international cooperation channels are blocked.
\end{abstract}

\tableofcontents
\newpage

% --- Strategic Context Section ---

\section{Strategic Context: Sanctions and Self-Reliance}

\subsection{Production Capacity and Global Position}

Iran ranks as the world's tenth-largest steel producer:

\textbf{2024 Production}: 31.4 million tonnes crude steel
\begin{itemize}[noitemsep]
\item Global rank: 10th
\item Middle East dominance: Largest producer in region
\item Per capita production: $\sim$360 kg (relatively high)
\item Export orientation: 25-30\% of production
\end{itemize}

\textbf{Technology distribution}:
\begin{itemize}[noitemsep]
\item DRI-EAF (natural gas-based): $\sim$55\% of capacity
\item Integrated BF-BOF: $\sim$35\% of capacity
\item Scrap EAF: $\sim$10\% of capacity
\item World's largest DRI production: $\sim$30 million tonnes annually
\end{itemize}

\textbf{Growth trajectory}:
\begin{itemize}[noitemsep]
\item Rapid expansion 2000-2020: From 6 MT to $>$30 MT
\item Government-promoted industrial development
\item Import substitution and export development strategy
\item Continued capacity expansion despite sanctions
\end{itemize}

\subsection{The Sanctions Environment}

\textbf{Comprehensive sanctions regime}:
\begin{itemize}[noitemsep]
\item US: Secondary sanctions targeting third parties
\item EU: Coordinated sanctions on financial, energy, technology sectors
\item UN: Periodic sanctions (currently reduced but monitoring continues)
\item Impact: Isolation from Western technology, financial systems, markets
\end{itemize}

\textbf{Effects on steel industry}:
\begin{itemize}[noitemsep]
\item Technology access: Limited to Chinese, Russian, domestic suppliers
\item Equipment procurement: Complex circumvention, higher costs, delays
\item Export markets: Restricted to friendly countries, complicated logistics
\item Financial transactions: Payment difficulties, currency restrictions
\item Knowledge exchange: Academic and technical isolation
\end{itemize}

\textbf{Sanctions circumvention and adaptation}:
\begin{itemize}[noitemsep]
\item Development of domestic equipment manufacturing capability
\item Technology partnerships with China, Russia, occasionally India
\item Barter arrangements and non-dollar trade
\item Front companies and indirect procurement channels
\item Reverse engineering and technology indigenization
\end{itemize}

\subsection{Natural Resource Advantages}

\textbf{Natural gas abundance}:
\begin{itemize}[noitemsep]
\item World's second-largest proven reserves (after Russia)
\item South Pars/North Dome field: Largest gas field globally (shared with Qatar)
\item Domestic gas prices: Heavily subsidized, very low cost
\item Stranded gas: Limited export infrastructure creates domestic surplus
\item Strategic advantage: Natural gas-based DRI economics highly favorable
\end{itemize}

\textbf{Iron ore resources}:
\begin{itemize}[noitemsep]
\item Significant domestic reserves (concentrated in central Iran)
\item Quality: Variable, generally moderate grade (50-60\% Fe)
\item Beneficiation industry developed to upgrade domestic ores
\item Some high-quality ore imports from neighbors when economically attractive
\end{itemize}

\textbf{Coal resources}:
\begin{itemize}[noitemsep]
\item Limited high-quality coking coal
\item Domestic coal generally unsuitable for steelmaking without blending
\item Historically imported coking coal (Australia, others) when possible
\item Sanctions complicating coal imports, favoring gas-based routes
\end{itemize}

% --- Additional sections omitted for brevity ---
% Use previous full version for content details as requested

% --- Conclusions Section ---

\section{Conclusions}

Iran's steel industry demonstrates how a middle-income country can develop substantial industrial capacity despite comprehensive international sanctions, leveraging natural resource advantages and state-directed development policy. The overwhelming reliance on natural gas-based DRI reflects both resource endowments and constraints imposed by isolation from global coal markets and technology suppliers.

\textbf{Key characteristics}:
\begin{itemize}[noitemsep]
\item Autarkic development driven by sanctions, not choice
\item Natural gas abundance enabling large-scale DRI production
\item State-coordinated industrial policy through IMIDRO
\item Technological self-reliance through necessity
\item Climate policy peripheral to economic and geopolitical imperatives
\end{itemize}

\textbf{Future trajectory}:

Iran's steel sector future depends overwhelmingly on geopolitical factors beyond industry control. Sanctions relief would enable technology leapfrogging and market expansion. Sanctions persistence means continued gradual development with widening technology gaps versus global leaders. In either scenario, decarbonization remains unlikely priority given development needs and absence of carbon pricing or international pressure mechanisms effective under sanctions.

The Iranian case offers lessons on industrial development under extreme constraints but is not a model other countries would choose to replicate. It exemplifies how resource endowments and political circumstances shape technological pathways more than optimal policy design.

\clearpage
\section*{References}

\begin{thebibliography}{99}

\bibitem{imidro2024}
IMIDRO (2024).
\textit{Iranian Steel Industry Annual Report}.
Tehran: IMIDRO.

\bibitem{mobarakeh2024}
Mobarakeh Steel Company (2024).
\textit{Annual Report 2023-2024}.
Isfahan: MSC.

\bibitem{worldsteel2024}
World Steel Association (2024).
\textit{World Steel in Figures 2024}.
Brussels: worldsteel.

\bibitem{iea_iran2023}
International Energy Agency (2023).
\textit{Iran Energy Statistics and Analysis}.
Paris: IEA.

\bibitem{iran_ndс2016}
Government of Iran (2016).
\textit{Intended Nationally Determined Contribution (INDC)}.
Tehran: Department of Environment.

\end{thebibliography}

\end{document}
